\section{Option Parsing}
\label{sec:optparse}

Option parsing is implemented using \textit{Pyhtons} \textit{optparse} (\cref{sec:depend}). The options are split into four groups.


\subsection{Input Files}
\label{subsec:inputFiles}

If the CSV files are not located in the program root directory a path can be specified using the following options.

\vspace{0.5em}
\begin{addmargin}[2em]{1em}
-N NODEFILE, --NodeFile=NODEFILE

\textit{CSV file with node definitions}

-S STRUTFILE, --StrutFile=STRUTFILE

\textit{CSV file with strut definitions}

-C CONSTFILE, --ConstraintFile=CONSTFILE

\textit{CSV file with constraint definitions}

-L STRUTLOADFILE, --StrutLoadFile=STRUTLOADFILE

\textit{CSV file with strut load definitions}

-F NODELOADFILE, --NodeLoadFile=NODELOADFILE

\textit{CSV file with node load definitions}
\end{addmargin}
\vspace{0.5em}

To generate templates with the column names from the type templates (\ref{ubsec:inputtemplates.py}) this option can be set and the program will exit after creating the files in the program root directory.
\vspace{0.5em}
\begin{addmargin}[2em]{1em}
-T 

\textit{Generate csv input template files}
\end{addmargin}
\vspace{0.5em}


\begin{inconsolata}
\begin{minipage}{\linewidth}
\begin{lstlisting}[language=python]
...
    parser = OptionParser(usage="usage: %prog [options]",
                          version="%prog " + str(version))

################################################################
#       Input files                                            #
################################################################

    group = OptionGroup(parser, "Input files")

    group.add_option("-N", "--NodeFile",
                     type="string",
                     action="store",
                     dest="nodeFile",
                     default='Input_nodes.csv',
                     help="CSV file with node definitions")

    group.add_option("-S", "--StrutFile",
                     type="string",
                     action="store",
                     dest="strutFile",
                     default='Input_struts.csv',
                     help="CSV file with strut definitions")

    group.add_option("-C", "--ConstraintFile",
                     type="string",
                     action="store",
                     dest="constFile",
                     default='Input_constraints.csv',
                     help="CSV file with constraint definitions")

    group.add_option("-L", "--StrutLoadFile",
                     type="string",
                     action="store",
                     dest="strutLoadFile",
                     default='Input_strutLoads.csv',
                     help="CSV file with strut load definitions")

    group.add_option("-F", "--NodeLoadFile",
                     type="string",
                     action="store",
                     dest="nodeLoadFile",
                     default='Input_nodeLoads.csv',
                     help="CSV file with node load definitions")

    group.add_option("-T",
                     action="store_true",
                     dest="genTemplates",
                     help="Generate csv input template files")

    parser.add_option_group(group)
...
\end{lstlisting}
\end{minipage}
\end{inconsolata}

\subsection{Output Files}
\label{subsec:outputFiles}

By default the CSV file containing the displacement vector $d$ will be written to the program root directory. A different path can be set with this option.

\vspace{2em}
\begin{addmargin}[2em]{1em}
-D DISPLACEMENTVECTORFILE,\\
--DisplacementVectorFile=DISPLACEMENTVECTORFILE 

\textit{Destination path to save the displacement vector csv file}
\end{addmargin}
\vspace{2em}

\begin{inconsolata}
\begin{minipage}{\linewidth}
\begin{lstlisting}[language=python]
...
################################################################
#       Output files                                           #
################################################################

    group = OptionGroup(parser, "Output files")

    group.add_option("-D", "--DisplacementVectorFile",
                     type="string",
                     action="store",
                     dest="displacementVectorFile",
                     default='displacement.csv',
                     help="Destination path to save the displacement vector csv file")

    parser.add_option_group(group)
...
\end{lstlisting}
\end{minipage}
\end{inconsolata}


\subsection{Graphics}
\label{subsec:graphics}

All elements of the visualization are programmatically generated with a scale factor. The default value of $2.0$ works well for most system geometries. However, for larger are smaller systems the scale factor can be adjusted.

\vspace{2em}
\begin{addmargin}[2em]{1em}
-s SCALE, --Scale=SCALE

\textit{Scales plot elements}
\end{addmargin}
\vspace{2em}

If the plot should not be displayed but saved as file a path can be set and the program will save the plot and exit instead of blocking.

\vspace{2em}
\begin{addmargin}[2em]{1em}
-P SAVEPLOT, --SavePlot=SAVEPLOT

\textit{Destination path to save the system plot. Supported filetypes: .png, .pdf}
\end{addmargin}
\vspace{2em}

\begin{inconsolata}
\begin{minipage}{\linewidth}
\begin{lstlisting}[language=python]
...
################################################################
#       graphics                                               #
################################################################

    group = OptionGroup(parser, "Graphics options")

    group.add_option("-s", "--Scale",
                     type="float",
                     action="store",
                     dest="scale",
                     default=2.0,
                     help="Scales plot elements")

    group.add_option("-P", "--SavePlot",
                     type="string",
                     action="store",
                     dest="savePlot",
                     default=False,
                     help="Destination path to save the system plot. Supported filetypes: .png, .pdf")

    parser.add_option_group(group)
...
\end{lstlisting}
\end{minipage}
\end{inconsolata}

\subsection{Debug}
\label{subsec:debug}

When the debug flag is set all element stiffness matrices and the global stiffness matrix will be printed to \textit{sys.stdout}.

\vspace{2em}
\begin{addmargin}[2em]{1em}
-d, --Debug

\textit{Print debug outputs}
\end{addmargin}
\vspace{2em}

\begin{inconsolata}
\begin{minipage}{\linewidth}
\begin{lstlisting}[language=python]
...
################################################################
#       debug                                                  #
################################################################

    group = OptionGroup(parser, "Debug options")

    group.add_option("-d", "--Debug",
                     action="store_true",
                     dest="debug",
                     help="Print debug outputs")

    parser.add_option_group(group)

################################################################
...
\end{lstlisting}
\end{minipage}
\end{inconsolata}

\subsection{Solver}
\label{subsec:solver}

The solver options allow it to specify which type of analysis is to be performed (first or second order theory) and threshold for the iteration can be set. 

At each iteration the delta in normal force is calculated. Iteration stops if this delta is below the threshold \textit{Epsilon}.

\vspace{2em}
\begin{addmargin}[2em]{1em}
-e, --Epsilon

\textit{Threshold at which to stop iteration}
\end{addmargin}
\vspace{2em}

This option enables the second order analysis.

\vspace{2em}
\begin{addmargin}[2em]{1em}
-2, --SecondOrder

\textit{Perform iterative second order analysis}
\end{addmargin}
\vspace{2em}

If the difference in deltas between iterations does not converge, the program iteration stops after an upper bound and prints an error message.

\vspace{2em}
\begin{addmargin}[2em]{1em}
-B, --iterBound

\textit{Number of iterations before giving up}
\end{addmargin}
\vspace{2em}

\begin{inconsolata}
\begin{minipage}{\linewidth}
\begin{lstlisting}[language=python]
...
################################################################
#       solver                                                 #
################################################################

    group = OptionGroup(parser, "Solver options")

    group.add_option("-e", "--Epsilon",
                                        type="float",
                                        action="store",
                                        dest="epsilon",
                                        default=10.0,
                                        help="Threshold at which to stop iteration")

    group.add_option("-2", "--SecondOrder",
                                        action="store_true",
                                        dest="secondOrder",
                                        help="Perform iterative second order analysis")

    group.add_option("-B", "--iterBound",
                                        type="int",
                                        action="store",
                                        dest="iterBound",
                                        default=1000,
                                        help="Number of iterations before giving up")

    parser.add_option_group(group)

################################################################
...
\end{lstlisting}
\end{minipage}
\end{inconsolata}

\pagebreak

\section{Input Parsing}
\label{sec:inputpars}

To parse the inputs (strut, node, load and constraint definitions) a set of templates is defined and a function generates a parser from that template.
The templates contain information about the structure of the CSV file, column names and types of the values.

\begin{inconsolata}
\begin{minipage}{\linewidth}
\begin{lstlisting}[language=python]
...
nodesTypeTemplate       = [ ('ID' , str),
                            ('X' , float),
                            ('Z' , float)]

strutsTypeTemplate      = [ ('ID' , str),
                            ('StartNode' , str),
                            ('StartHinge' , int),
                            ('EndNode' , str),
                            ('EndHinge' , int),
                            ('E' , float),
                            ('A' , float),
                            ('I' , float)]

constraintTypeTemplate  = [ ('Node' , str),
                            ('x' , int),
                            ('z' , int),
                            ('r' , int)]

strutLoadTypeTemplate   = [ ('Strut' , str),
                            ('Type' , int),
                            ('x1' , float),
                            ('x2' , float),
                            ('q' , float),
                            ('F' , float),
                            ('M' , float)]

nodeLoadTypeTemplate    = [ ('Node' , str),
                            ('Fx' , float),
                            ('Fz' , float),
                            ('M' , float)]
...
\end{lstlisting}
\end{minipage}
\end{inconsolata}

\pagebreak

\subsection{CSV Parser}
\label{sec:csvparse}

The parser reads a CSV file using \textit{Pythons optparse} (\ref{sec:depend}) and checks if each row contains an entry for every column.

\begin{inconsolata}
\begin{minipage}{\linewidth}
\begin{lstlisting}[language=python]
...
def readCSV(csvFile, typeTemplate):
    ID = 0
    with open(csvFile) as csvfile:
        _dict = {}
        reader = csv.DictReader(csvfile, delimiter = ';')

        for row in reader:

            if len(row) > len(typeTemplate):
                print('Too many entries in row ' + str(ID + 1) + ': ' + csvFile)
                exit()

            for key, value in row.iteritems():
                if value is None:
                    print('Too few entries in row ' + str(ID + 1) + ': ' + csvFile)
                    exit()
...
\end{lstlisting}
\end{minipage}
\end{inconsolata}

Then it tries to cast each entry from type \textit{string} to the type specified in the template.

\begin{inconsolata}
\begin{minipage}{\linewidth}
\begin{lstlisting}[language=python]
...
            row = typeConv(row, typeTemplate)
            _dict[ID] = row
            ID += 1

    return _dict
...
\end{lstlisting}
\end{minipage}
\end{inconsolata}

If casting fails the program exits with an error message indicating which input string is malformed.

\begin{inconsolata}
\begin{minipage}{\linewidth}
\begin{lstlisting}[language=python]
...
def typeConv(_dict, _typeTemplate):
    _typeTemplate = dict(_typeTemplate)
    for key, value in _dict.iteritems():
        try:
            _dict[key] = _typeTemplate[key](value)
        except ValueError:
            print('The ' + key + ' value is not of type ' + str(_typeTemplate[key]) + '.')
            exit()

    return _dict
...
\end{lstlisting}
\end{minipage}
\end{inconsolata}

This method uses \textit{Pythons} casting methods, which are very sensible and reliable \cite[2.4.1-2.4.6]{pythonTypes}.
The \textit{input.py} (\cref{subsec:input.py}) also contains a function to generate CSV template files from the type templates, making it very convenient to change the data structure by changing the type template and calling \textit{python ./structana/main.py -T} to generate a new set of CSV templates.

\begin{inconsolata}
\begin{minipage}{\linewidth}
\begin{lstlisting}[language=python]
...
def genTemplate(filename, typeTemplate):
    fieldnames = []
    for key, value in typeTemplate:
        fieldnames.append(key)
    with open(filename, 'w') as csvfile:
        writer = csv.writer(csvfile, delimiter = ';')
        writer.writerow(fieldnames)
...
\end{lstlisting}
\end{minipage}
\end{inconsolata}

\subsection{Input Checking}
\label{sec:inputcheck}

After parsing \textit{node} and \textit{strut} definitions nodes which aren't referenced by any struts (free nodes) are deleted. 

\begin{inconsolata}
\begin{minipage}{\linewidth}
\begin{lstlisting}[language=python]
...
def deleteFreeNodes(nodes, struts):
    freeNodes = []
    for ID, node in nodes.iteritems():
        isFreeNode = True
        for strutID, strut in struts.iteritems():
            if (strut['StartNode'] == nodeIDtoName(ID, nodes)) or (strut['EndNode'] == nodeIDtoName(ID, nodes)):
                isFreeNode = False
                pass
        if isFreeNode:
            print('Node ' + str(nodeIDtoName(ID, nodes)) + ' is a free node and will be deleted.')
            freeNodes.append(ID)

    for freeNode in freeNodes:
        del nodes[freeNode]
...
\end{lstlisting}
\end{minipage}
\end{inconsolata}

Then some logical checks are performed.

\pagebreak

\subsubsection{Strut Node References}
\label{sec:inputcheck-checkStrutNodes}

\textit{checkStrutNodes} checks if the nodes referenced by the struts start- and end-node are defined.

\begin{inconsolata}
\begin{minipage}{\linewidth}
\begin{lstlisting}[language=python]
...
def checkStrutNodes(nodes, struts):
    #assemble node IDs form dict
    NodeIDs = []
    for ID, node in nodes.iteritems():
        NodeIDs.append(node['ID'])

    for ID, strut in struts.iteritems():
        if strut['StartNode'] not in NodeIDs:
            print('Node ' + str(strut['StartNode']) + ' referenced by strut ' + str(strut['ID']) + ' but not defined.')
            exit()
        if strut['EndNode'] not in NodeIDs:
            print('Node ' + str(strut['EndNode']) + ' referenced by strut ' + str(strut['ID']) + ' but not defined.')
            exit()
...
\end{lstlisting}
\end{minipage}
\end{inconsolata}

\subsubsection{Constraint Node References}
\label{sec:inputcheck-checkConstraintNodes}

\textit{checkConstraintNodes} checks if the nodes referenced by the constraints are defined.

\begin{inconsolata}
\begin{minipage}{\linewidth}
\begin{lstlisting}[language=python]
...
def checkConstraintNodes(nodes, constraints):
    #assemble node IDs form dict
    NodeIDs = []
    for ID, node in nodes.iteritems():
        NodeIDs.append(node['ID'])

    for ID, const in constraints.iteritems():
        if const['Node'] not in NodeIDs:
            print('Node ' + str(const['Node']) + ' referenced by constraint ' + str(ID) + ' but not defined.')
            exit()
...
\end{lstlisting}
\end{minipage}
\end{inconsolata}

\pagebreak

\subsubsection{Strut Load References}
\label{sec:inputcheck-checkStrutLoads}

\textit{checkStrutLoads} checks if the struts referenced by the strut loads are defined.

\begin{inconsolata}
\begin{minipage}{\linewidth}
\begin{lstlisting}[language=python]
...
def checkStrutLoads(struts, strutLoads):
    #assemble strut IDs form dict
    StrutIDs = []
    for ID, strut in struts.iteritems():
        StrutIDs.append(strut['ID'])

    for ID, load in strutLoads.iteritems():
        if load['Strut'] not in StrutIDs:
            print('Strut ' + str(load['Strut']) + ' referenced by load ' + str(ID) + ' but not defined.')
            exit()
...
\end{lstlisting}
\end{minipage}
\end{inconsolata}

\subsubsection{Node Load References}
\label{sec:inputcheck-checkNodeLoads}

\textit{checkNodeLoads} checks if the nodes referenced by the node loads are defined.

\begin{inconsolata}
\begin{minipage}{\linewidth}
\begin{lstlisting}[language=python]
...
def checkNodeLoads(nodes, nodeLoads):
    #assemble node IDs form dict
    NodeIDs = []
    for ID, node in nodes.iteritems():
        NodeIDs.append(node['ID'])

    for ID, load in nodeLoads.iteritems():
        if load['Node'] not in NodeIDs:
            print('Node ' + str(load['Node']) + ' referenced by load ' + str(ID) + ' but not defined.')
            exit()
...
\end{lstlisting}
\end{minipage}
\end{inconsolata}

