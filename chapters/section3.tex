\section{Assembling Matrices and Vectors}
\label{sec:asmmatrvec}

In order to solve the system all vectors and matrices need to be assembled and the system checked for non-trivial solutions.
From the solution the member forces (stress resultants) can be calculated.

\subsection{Global Member Force Vector $S\textsubscript{G}$}
\label{sec:asmSGZ}

The global member force vector $S\textsubscript{G}$ is constructed by first assembling the local member forces from the strut loads.

\begin{inconsolata}
\begin{minipage}{\linewidth}
\begin{lstlisting}[language=python]
...
#adds up the load vectors and adds them to the strut dict
def assemble_S_L( strutLoads, struts, nodes ):
    #assemble strut IDs form dict
    StrutIDs = {}
    for ID, strut in struts.iteritems():
        StrutIDs[strut['ID']] = ID

    S_L = {}
    for ID, load in strutLoads.iteritems():
        strut = struts[StrutIDs[load['Strut']]]
        id = StrutIDs[strut['ID']]
        if id in S_L:
            S_L[id] += get_S_L(load, strut)
        else:
            S_L[id] = get_S_L(load, strut)

    for ID, v in S_L.iteritems():
        struts[ID]['S_L'] = v
...
\end{lstlisting}
\end{minipage}
\end{inconsolata}

Then the vector is initiated with zeroes and a length $n \cdot 3$ with $n$ being the number of nodes.

\begin{inconsolata}
\begin{minipage}{\linewidth}
\begin{lstlisting}[language=python]
...
#creates the global load vector from strut loads and node loads
def assemble_S_G( nodeLoads, struts, nodes ):
    size = len(nodes) * 3
    S_G = np.zeros(size)
\end{lstlisting}
\end{minipage}
\end{inconsolata}

The node loads are entered in global coordinates and can be added directly to the vector.

\begin{inconsolata}
\begin{minipage}{\linewidth}
\begin{lstlisting}[language=python]
    #add node loads
    for ID, load in nodeLoads.iteritems():
        id = nodeNameToID(load['Node'], nodes)
        Fx = load['Fx']
        Fz = load['Fz']
        M  = load['M']
        v = np.array([Fx, Fz, M])
        for i in range(3):
            S_G[id*3 + i] += v[i]
\end{lstlisting}
\end{minipage}
\end{inconsolata}

The local member forces need to be transformed to global coordinates ($rot(\alpha) \boldsymbol{\cdot} S_L$) before they can be added.

\begin{inconsolata}
\begin{minipage}{\linewidth}
\begin{lstlisting}[language=python]
    #add strut load vectors
    for ID, strut in struts.iteritems():
        id = nodeNameToID(strut['StartNode'], nodes)

        S_L = np.zeros(6)
        if 'S_L' in strut:
            S_L = strut['S_L']

        #rotate to global coordinates
        alpha = strut['alpha']
        v = rot(alpha).dot(S_L)
        for i in range(6):
            S_G[id*3 + i] += v[i] * -1

    return S_G
...
\end{lstlisting}
\end{minipage}
\end{inconsolata}

\pagebreak

\subsection{System Stiffness Matrix $K$}
\label{sec:asmK}

The system stiffness matrix is assembled by inserting the global strut stiffness matrices and adding the components where the same node is referenced.
Suppose the system were defined by four nodes ($n=4$) and three struts with strut 1 from node 0 to 1, strut 2 from node 1 to 2 and strut 3 from node 2 to 3.

\begin{equation} \label{strut1}
    K_{\textcolor{solarized-red}{1}} = \begin{bmatrix}
        k_{1,1}  & k_{1,2} & k_{1,3}  & k_{1,4}   & k_{1,5}   & k_{1,6} \\
                 & k_{2,2} & k_{2,3}  & k_{2,4}   & k_{2,5}   & k_{2,6} \\
                 &         & k_{3,3}  & k_{3,4}   & k_{3,5}   & k_{3,6} \\
                 &         &          & k_{4,4}   & k_{4,5}   & k_{4,6} \\
                 &         &          &           & k_{5,5}   & k_{5,6} \\
        Sym.     &         &          &           &           & k_{6,6}   
    \end{bmatrix}
\end{equation}
\begin{equation} \label{strut2}
    K_{\textcolor{solarized-cyan}{2}} = \begin{bmatrix}
        k_{1,1}  & k_{1,2} & k_{1,3}  & k_{1,4}   & k_{1,5}   & k_{1,6} \\
                 & k_{2,2} & k_{2,3}  & k_{2,4}   & k_{2,5}   & k_{2,6} \\
                 &         & k_{3,3}  & k_{3,4}   & k_{3,5}   & k_{3,6} \\
                 &         &          & k_{4,4}   & k_{4,5}   & k_{4,6} \\
                 &         &          &           & k_{5,5}   & k_{5,6} \\
        Sym.     &         &          &           &           & k_{6,6}   
    \end{bmatrix}
\end{equation}
\begin{equation} \label{strut3}
    K_{\textcolor{solarized-green}{3}} = \begin{bmatrix}
        k_{1,1}  & k_{1,2} & k_{1,3}  & k_{1,4}   & k_{1,5}   & k_{1,6} \\
                 & k_{2,2} & k_{2,3}  & k_{2,4}   & k_{2,5}   & k_{2,6} \\
                 &         & k_{3,3}  & k_{3,4}   & k_{3,5}   & k_{3,6} \\
                 &         &          & k_{4,4}   & k_{4,5}   & k_{4,6} \\
                 &         &          &           & k_{5,5}   & k_{5,6} \\
        Sym.     &         &          &           &           & k_{6,6}   
    \end{bmatrix}
\end{equation}

The system stiffness matrix would be composed as shown in \cref{fullstiffnessmatrix}.

\begin{inconsolata}
\begin{minipage}{\linewidth}
\begin{lstlisting}[language=python]
...
def insert( M, K, start_x, start_y, fn ):
    size_x, size_y = K.shape

    if (start_x + size_x > M.shape[0]) or (start_y + size_y > M.shape[1]):
        print "Size mismatch inserting K into M at (" + str(start_x) + "," + str(start_y) + ")."
        exit()

    for y in range(size_y):
        for x in range(size_x - y):
            M[start_y + y][start_x + x + y] = fn(M[start_y + y][start_x + x + y], K[y][x + y])
    return M
...
\end{lstlisting}
\end{minipage}
\end{inconsolata}

This function performs the operation visualized in (\cref{fullstiffnessmatrix}).
The indices are computed such that the operations are only performed on the lower left of the matrices.
It takes a function as an argument \textit{fn} that is applied to two values (the value already present and that of the matrix to be inserted) and the result is inserted into the matrix. 

\newcommand{\cb}[1]{\textcolor{solarized-red}{#1}}
\newcommand{\cc}[1]{\textcolor{solarized-cyan}{#1}}
\newcommand{\cg}[1]{\textcolor{solarized-green}{#1}}

\newcommand{\tikzmarkb}[1]{\tikz[overlay,remember picture] \node (#1) {};}
\newcommand{\DrawBoxb}[2][]{%
    \tikz[overlay,remember picture]{
    \draw[solarized-red,#1]
      ($(left1)+(-0.2em,0.6em)$) rectangle
      ($(right1)+(0.1em,-0.3em)$);
    \draw[solarized-cyan,#1]
      ($(left2)+(-0.2em,0.6em)$) rectangle
      ($(right2)+(0.1em,-0.3em)$);
    \draw[solarized-green,#1]
      ($(left3)+(-0.2em,0.6em)$) rectangle
      ($(right3)+(0.1em,-0.3em)$);}
}

\subsection{Applying Constraints}
\label{sec:applyconst}

After assembly the constraints are applied. This could be achieved by deleting the rows and columns which represent the translations inhibited by the constraints.
But that would require to keep track of the indices. Instead the rows and columns are set to zero and the coefficient on the diagonal to one.

\begin{inconsolata}
\begin{minipage}{\linewidth}
\begin{lstlisting}[language=python]
...
def apply_constraints(K, struts, nodes, constraints):
    #apply constraints
    size = len(nodes) * 3
    zero = np.zeros(size)
    for ID, const in constraints.iteritems():
        id = nodeNameToID(const['Node'], nodes)
        if const['x']:
            x = id * 3
            K[:,x] = zero
            K[x,:] = zero
        if const['z']:
            x = id * 3 + 1
            K[:,x] = zero
            K[x,:] = zero
        if const['r']:
            x = id * 3 + 2
            K[:,x] = zero
            K[x,:] = zero

    #check for zeros in column and row
    for i in range(len(K[0])):
        if np.all(K[:,i] == 0) and np.all(K[i,:] == 0):
            K[i][i] = 1
...
\end{lstlisting}
\end{minipage}
\end{inconsolata}

\section{Checking for Kinematic System Conditions}
\label{sec:kinesyscheck}

Two criteria are checked. First if at least three constraints are defined.

\begin{equation}
n < 3
\end{equation}

And then if the system has any non-trivial solutions.

\begin{equation}
det \lvert K \lvert = 0
\end{equation}

\pagebreak

\section{Solving the Resulting System of Linear Equations}
\label{sec:solver}

With the global load vector $S_G$ and the system stiffness matrix $K$ the system (\cref{systemeq}) can be solved for the deflection vector $d$ ($S_G = K \cdot d$).
With $n$ being the number of nodes:

\begin{equation} \label{systemeq}
    \begin{pmatrix}
        F_{1}^x \\
        F_{1}^y \\
        M_{1} \\
        \vdots \\
        F_{n}^x \\
        F_{n}^y \\
        M_{n}
    \end{pmatrix} = \begin{bmatrix}
        k_{1,1}  & k_{1,2} & k_{1,3}  & \cdots & k_{1,n-2}   & k_{1,n-1}   & k_{1,n}   \\
                 & k_{2,2} & k_{2,3}  & \cdots & k_{2,n-2}   & k_{2,n-1}   & k_{2,n}   \\
                 &         & k_{3,3}  & \cdots & k_{3,n-2}   & k_{3,n-1}   & k_{3,n}   \\
                 &         &          & \ddots & \vdots      & \vdots      & \vdots    \\
                 &         &          &        & k_{n-2,n-2} & k_{n-2,n-1} & k_{n-2,n} \\
                 &         &          &        &             & k_{n-1,n-1} & k_{n-1,n} \\
        Sym.     &         &          &        &             &             & k_{n,n}   
    \end{bmatrix} \cdot \begin{pmatrix}
        d_{1}^x \\
        d_{1}^y \\
        d_{1}^r \\
        \vdots \\
        d_n^x \\
        d_n^y \\
        d_n^r
    \end{pmatrix}
\end{equation}

\section{Calculate Member Forces}
\label{sec:calcmemberforces}

After obtaining the solution the local member forces can be calculated by solving $S_l = K_n \cdot d$ where $K_n$ is the member stiffness matrix and $d$ the deflection vector with the components corresponding to the respective member.

\begin{inconsolata}
\begin{minipage}{\linewidth}
\begin{lstlisting}[language=python]
...
# calculate the local strut forces
def calc_local_forces(nodes, struts, d):
    for ID, strut in struts.iteritems():
        alpha = strut['alpha']
        K = strut["K"]
        r = rot(alpha)

        x1 = nodeNameToID(strut["StartNode"], nodes) * 3
        x2 = x1 + 3
        x3 = nodeNameToID(strut["EndNode"], nodes) * 3
        x4 = x3 + 3

        dl = np.append(d[x1:x2], d[x3:x4]*-1)

        Sg = np.dot(K, dl)
        Sl = np.dot(r, Sg)

        strut["Sl"] = Sl
...
\end{lstlisting}
\end{minipage}
\end{inconsolata}

\pagebreak
\begin{sidewaysfigure}

\begin{equation} \label{fullstiffnessmatrix}
    K = \begin{bmatrix}\begin{smallmatrix}
\tikzmarkb{left1}k_{\cb{1}(1,1)} & k_{\cb{1}(1,2)} & k_{\cb{1}(1,3)}  & k_{\cb{1}(1,4)}                   & k_{\cb{1}(1,5)}                    & k_{\cb{1}(1,6)}                                     & 0                                 & 0                                 & 0                                 & 0               & 0               & 0 \\
                                 & k_{\cb{1}(2,2)} & k_{\cb{1}(2,3)}  & k_{\cb{1}(2,4)}                   & k_{\cb{1}(2,5)}                    & k_{\cb{1}(2,6)}                                     & 0                                 & 0                                 & 0                                 & 0               & 0               & 0 \\
                                 &                 & k_{\cb{1}(3,3)}  & k_{\cb{1}(3,4)}                   & k_{\cb{1}(3,5)}                    & k_{\cb{1}(3,6)}                                     & 0                                 & 0                                 & 0                                 & 0               & 0               & 0 \\
                                 &                 &                  & \tikzmarkb{left2}k_{\cb{1}(4,4)} + k_{\cc{2}(1,1)} & k_{\cb{1}(4,5)} + k_{\cc{2}(1,2)}  & k_{\cb{1}(4,6)} + k_{\cc{2}(1,3)}                   & k_{\cc{2}(1,4)}                   & k_{\cc{2}(1,5)}                   & k_{\cc{2}(1,6)}                   & 0               & 0               & 0 \\
                                 &                 &                  &                                   & k_{\cb{1}(5,5)} + k_{\cc{2}(2,2)}  & k_{\cb{1}(5,6)} + k_{\cc{2}(2,3)}                   & k_{\cc{2}(2,4)}                   & k_{\cc{2}(2,5)}                   & k_{\cc{2}(2,6)}                   & 0               & 0               & 0 \\
                                 &                 &                  &                                   &                                    & k_{\cb{1}(6,6)} + k_{\cc{2}(3,3)}\tikzmarkb{right1} & k_{\cc{2}(3,4)}                   & k_{\cc{2}(3,5)}                   & k_{\cc{2}(3,6)}                   & 0               & 0               & 0 \\
                                 &                 &                  &                                   &                                    &                                                     & \tikzmarkb{left3}k_{\cc{2}(4,4)} + k_{\cg{3}(1,1)} & k_{\cc{2}(4,5)} + k_{\cg{3}(1,2)} & k_{\cc{2}(4,6)} + k_{\cg{3}(1,3)} & k_{\cg{3}(1,4)} & k_{\cg{3}(1,5)} & k_{\cg{3}(1,6)} \\
                                 &                 &                  &                                   &                                    &                                                     &                                   & k_{\cc{2}(5,5)} + k_{\cg{3}(2,2)} & k_{\cc{2}(5,6)} + k_{\cg{3}(2,3)} & k_{\cg{3}(2,4)} & k_{\cg{3}(2,5)} & k_{\cg{3}(2,6)} \\
                                 &                 &                  &                                   &                                    &                                                     &                                   &                                   & k_{\cc{2}(6,6)} + k_{\cg{3}(3,3)}\tikzmarkb{right2} & k_{\cg{3}(3,4)} & k_{\cg{3}(3,5)} & k_{\cg{3}(3,6)} \\
                                 &                 &                  &                                   &                                    &                                                     &                                   &                                   &                                   & k_{\cg{3}(4,4)} & k_{\cg{3}(4,5)} & k_{\cg{3}(4,6)} \\
                                 &                 &                  &                                   &                                    &                                                     &                                   &                                   &                                   &                 & k_{\cg{3}(5,5)} & k_{\cg{3}(5,6)} \\
Sym.                             &                 &                  &                                   &                                    &                                                     &                                   &                                   &                                   &                 &                 & k_{\cg{3}(6,6)}\tikzmarkb{right3}
    \end{smallmatrix}\end{bmatrix}
\end{equation}
\DrawBoxb[thick]

\end{sidewaysfigure}
\pagebreak